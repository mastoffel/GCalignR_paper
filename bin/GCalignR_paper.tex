\documentclass[]{article}
\usepackage{lmodern}
\usepackage{amssymb,amsmath}
\usepackage{ifxetex,ifluatex}
\usepackage{fixltx2e} % provides \textsubscript
\ifnum 0\ifxetex 1\fi\ifluatex 1\fi=0 % if pdftex
  \usepackage[T1]{fontenc}
  \usepackage[utf8]{inputenc}
\else % if luatex or xelatex
  \ifxetex
    \usepackage{mathspec}
  \else
    \usepackage{fontspec}
  \fi
  \defaultfontfeatures{Ligatures=TeX,Scale=MatchLowercase}
\fi
% use upquote if available, for straight quotes in verbatim environments
\IfFileExists{upquote.sty}{\usepackage{upquote}}{}
% use microtype if available
\IfFileExists{microtype.sty}{%
\usepackage{microtype}
\UseMicrotypeSet[protrusion]{basicmath} % disable protrusion for tt fonts
}{}
\usepackage[margin=1in]{geometry}
\usepackage{hyperref}
\hypersetup{unicode=true,
            pdftitle={GCalignR: An R package for aligning Gas-Chromatography data},
            pdfauthor={Meinolf Ottensmann, Martin A. Stoffel, Barbara Caspers, Joseph I. Hoffman},
            pdfborder={0 0 0},
            breaklinks=true}
\urlstyle{same}  % don't use monospace font for urls
\usepackage{color}
\usepackage{fancyvrb}
\newcommand{\VerbBar}{|}
\newcommand{\VERB}{\Verb[commandchars=\\\{\}]}
\DefineVerbatimEnvironment{Highlighting}{Verbatim}{commandchars=\\\{\}}
% Add ',fontsize=\small' for more characters per line
\newenvironment{Shaded}{}{}
\newcommand{\KeywordTok}[1]{\textbf{{#1}}}
\newcommand{\DataTypeTok}[1]{\textcolor[rgb]{0.50,0.00,0.00}{{#1}}}
\newcommand{\DecValTok}[1]{\textcolor[rgb]{0.00,0.00,1.00}{{#1}}}
\newcommand{\BaseNTok}[1]{\textcolor[rgb]{0.00,0.00,1.00}{{#1}}}
\newcommand{\FloatTok}[1]{\textcolor[rgb]{0.50,0.00,0.50}{{#1}}}
\newcommand{\ConstantTok}[1]{\textcolor[rgb]{0.00,0.00,0.00}{{#1}}}
\newcommand{\CharTok}[1]{\textcolor[rgb]{1.00,0.00,1.00}{{#1}}}
\newcommand{\SpecialCharTok}[1]{\textcolor[rgb]{1.00,0.00,1.00}{{#1}}}
\newcommand{\StringTok}[1]{\textcolor[rgb]{0.87,0.00,0.00}{{#1}}}
\newcommand{\VerbatimStringTok}[1]{\textcolor[rgb]{0.87,0.00,0.00}{{#1}}}
\newcommand{\SpecialStringTok}[1]{\textcolor[rgb]{0.87,0.00,0.00}{{#1}}}
\newcommand{\ImportTok}[1]{{#1}}
\newcommand{\CommentTok}[1]{\textcolor[rgb]{0.50,0.50,0.50}{\textit{{#1}}}}
\newcommand{\DocumentationTok}[1]{\textcolor[rgb]{0.50,0.50,0.50}{\textit{{#1}}}}
\newcommand{\AnnotationTok}[1]{\textcolor[rgb]{0.50,0.50,0.50}{\textbf{\textit{{#1}}}}}
\newcommand{\CommentVarTok}[1]{\textcolor[rgb]{0.50,0.50,0.50}{\textbf{\textit{{#1}}}}}
\newcommand{\OtherTok}[1]{{#1}}
\newcommand{\FunctionTok}[1]{\textcolor[rgb]{0.00,0.00,0.50}{{#1}}}
\newcommand{\VariableTok}[1]{{#1}}
\newcommand{\ControlFlowTok}[1]{{#1}}
\newcommand{\OperatorTok}[1]{{#1}}
\newcommand{\BuiltInTok}[1]{{#1}}
\newcommand{\ExtensionTok}[1]{{#1}}
\newcommand{\PreprocessorTok}[1]{\textbf{{#1}}}
\newcommand{\AttributeTok}[1]{{#1}}
\newcommand{\RegionMarkerTok}[1]{{#1}}
\newcommand{\InformationTok}[1]{\textcolor[rgb]{0.50,0.50,0.50}{\textbf{\textit{{#1}}}}}
\newcommand{\WarningTok}[1]{\textcolor[rgb]{1.00,0.00,0.00}{\textbf{{#1}}}}
\newcommand{\AlertTok}[1]{\textcolor[rgb]{0.00,1.00,0.00}{\textbf{{#1}}}}
\newcommand{\ErrorTok}[1]{\textcolor[rgb]{1.00,0.00,0.00}{\textbf{{#1}}}}
\newcommand{\NormalTok}[1]{{#1}}
\usepackage{graphicx,grffile}
\makeatletter
\def\maxwidth{\ifdim\Gin@nat@width>\linewidth\linewidth\else\Gin@nat@width\fi}
\def\maxheight{\ifdim\Gin@nat@height>\textheight\textheight\else\Gin@nat@height\fi}
\makeatother
% Scale images if necessary, so that they will not overflow the page
% margins by default, and it is still possible to overwrite the defaults
% using explicit options in \includegraphics[width, height, ...]{}
\setkeys{Gin}{width=\maxwidth,height=\maxheight,keepaspectratio}
\IfFileExists{parskip.sty}{%
\usepackage{parskip}
}{% else
\setlength{\parindent}{0pt}
\setlength{\parskip}{6pt plus 2pt minus 1pt}
}
\setlength{\emergencystretch}{3em}  % prevent overfull lines
\providecommand{\tightlist}{%
  \setlength{\itemsep}{0pt}\setlength{\parskip}{0pt}}
\setcounter{secnumdepth}{5}
% Redefines (sub)paragraphs to behave more like sections
\ifx\paragraph\undefined\else
\let\oldparagraph\paragraph
\renewcommand{\paragraph}[1]{\oldparagraph{#1}\mbox{}}
\fi
\ifx\subparagraph\undefined\else
\let\oldsubparagraph\subparagraph
\renewcommand{\subparagraph}[1]{\oldsubparagraph{#1}\mbox{}}
\fi

%%% Use protect on footnotes to avoid problems with footnotes in titles
\let\rmarkdownfootnote\footnote%
\def\footnote{\protect\rmarkdownfootnote}

%%% Change title format to be more compact
\usepackage{titling}

% Create subtitle command for use in maketitle
\newcommand{\subtitle}[1]{
  \posttitle{
    \begin{center}\large#1\end{center}
    }
}

\setlength{\droptitle}{-2em}
  \title{GCalignR: An R package for aligning Gas-Chromatography data}
  \pretitle{\vspace{\droptitle}\centering\huge}
  \posttitle{\par}
  \author{Meinolf Ottensmann, Martin A. Stoffel, Barbara Caspers, Joseph I.
Hoffman}
  \preauthor{\centering\large\emph}
  \postauthor{\par}
  \predate{\centering\large\emph}
  \postdate{\par}
  \date{2017-01-09}


\begin{document}
\maketitle

\section{Abstract}\label{abstract}

Key-words: GC-MS,gas-chromatography, chemical communication, olfactory
communication, alignment

\subsection{Introduction}\label{introduction}

Chemical cues are arguably the most common mode of communication in
animals (Wyatt 2014), where the role is evident in kin (Porter and Moore
1981; Krause et al. 2012; Gilad et al. 2016) and mate recognition
(Martin, Helanterä, and Drijfhout 2008), mate-choice (Penn 2002) and
signalling of genetic quality (Charpentier et al. 2010; Stoffel et al.
2015). The investigation of chemical signatures requires the use of
metabolomic approaches in order to characterise and compare the involved
chemicals. Gas-chromatography is a widespread analytical method to
unravel the compostition of samples with high efficiency (McNair and
Miller 2011). For the detection of broader patterns in chemical samples
researchers use an untargeted approach and analyse the whole spectrum of
sampled chemicals rather than targeting specific compounds. However,
chromatography data across multiple samples are not directly comparable
as they need to be aligned first. Furthermore, the retention times of
peaks vary across samples due to subtle, random and often unavoidable
variation of the chromatograph machine parameters (Pierce et al. 2005
and references within). For studies that seek to identify chemical
patterns across samples it becomes essential to account for these
retention time drifts by using an appropriate alignment method. A number
of automated tools is available for two-dimensional chromatograms
(LC-MS, GC-MS) which offer mass spectra in addition to retention times
(Pierce et al. 2005; C. A. Smith et al. 2006; Robinson et al. 2007;
Luedemann et al. 2008; Koh et al. 2010; Jiang et al. 2013; Zhang et al.
2012; Niu et al. 2014; R. Smith, Ventura, and Prince 2015), while there
is to our knowledge only one application proposed for the needs of
one-dimensional chromatograms (Dellicour and Lecocq 2013). As a
consequence, many researchers rely on a manual alignment of their data
(Charpentier et al. 2010; Caspers et al. 2011; Harris, Davies, and Nicol
2012); (Citations are critical. Because nothing is mentioned on
alignment within papers, cf.~\url{http://bit.ly/2gYJAZw}). This approach
bears three main drawbacks: (1) In large scale studies the task becomes
a difficult and time consuming task. (2) Humans are prone to detect
patterns in noise which is why the researcher may bias the alignment due
to subjective experience and expectations. (3) The data analytic
pipeline from the raw gas-chromatography data to the results of the
statistical analysis is not reproducible. Here, we introduce
\texttt{GCalignR}, a package developed in the language \texttt{R} (R
Core Team 2016), which provides a simple and fast algorithm to align
peaks from GC data and evaluate the alignment of empirical
data.\texttt{GCalignR} was specifically developed and tested as a
preprocessing tool prior to the statistical analysis of chemical samples
from animal skin and preen glands (see Stoffel et al. (2015) for an
application of the underlying algorithm). The main focus of the package
is set on the alignment of homologous retention times and the inspection
of the resulting data. The algorithm consists of two main steps: (1)
Systematic shifts of chromatograms are corrected by applying appropriate
linear shifts to whole chromatograms based on a single reference. (2)
Retention times of individual peaks are step-wise grouped together with
homologous peaks of other samples within one row. The outcome of this
grouping procedures can be altered by specifying three parameters that
are described below. Among several optional processing steps, the
package allows to remove peaks belonging to contaminations, which are
identified due to there presence in control samples. Furthermore we
demonstrate the easy integration of the R-packge \texttt{vegan}(Oksanen
et al. 2016) into a solid workflow for multivariate analyses of chemical
data, which can be fully integrated into in \texttt{Rmarkdown} documents
(Allaire et al. 2016) to fullfil good standards of reproducibility (Peng
2011).

\subsection{Material and Methods}\label{material-and-methods}

\paragraph{maybe a flowdiagram (package DiagramR) to illustrate the
complete
workflow}\label{maybe-a-flowdiagram-package-diagramr-to-illustrate-the-complete-workflow}

\begin{enumerate}
\def\labelenumi{\arabic{enumi}.}
\tightlist
\item
  GC / GC-MS analysis
\item
  Peak detection software
\item
  GCalignR workflow
\item
  statistical analysis
\end{enumerate}

\subsection{Input data}\label{input-data}

Peaks have to be extracted from raw chromatogram files using proprietary
or freew software prior to alignment with GCalignR. The standard working
format of \texttt{GCalignR}is text file containing peak retention times
and a arbitrary number of further variables (see Supporting information
A). Additionally the use of \texttt{lists} in R is supported.
\texttt{GCalignR} aligns peaks via their retention times (and not their
mass-spectra, which may not be available, e.g.~when using
gas-chromatography coupled to a flame ionization detector (FID)) to
align the peaks across individuals for subsequent chemometric analysis
and pattern detection .The simple assumption is that peaks with similar
retention times represent the same substances. However, it is highly
recommended to verify this assumption by comparing also the mass-spectra
(if available) of the substances of interest. The final data is returned
either in form of a list within R or saved as text files (.txt)

\section{Example dataset}\label{example-dataset}

\paragraph{explanation of example
dataset}\label{explanation-of-example-dataset}

\section{GCalignR workflow}\label{gcalignr-workflow}

\begin{itemize}
\tightlist
\item
  GCalignR steps: Checking the input, aligning chromatograms, evaluating
  alignment
\item
  adjust parameters, align again, evaluate again (if first alignment
  wasn´t satisfactory)
\end{itemize}

\section{Input}\label{input}

\begin{itemize}
\tightlist
\item
  Quickly describe input formats
\item
  Check input and what it checks
\end{itemize}

\begin{Shaded}
\begin{Highlighting}[]
\KeywordTok{check_input}\NormalTok{(}\DataTypeTok{data =} \NormalTok{peak_data,}\DataTypeTok{show_peaks =} \NormalTok{T, }\DataTypeTok{col=} \StringTok{"red"}\NormalTok{) }\CommentTok{# If show_peaks = T, a histogram of peaks is plotted }
\end{Highlighting}
\end{Shaded}

\section{Aligning peaks}\label{aligning-peaks}

\begin{itemize}
\tightlist
\item
  describe main features of the main function
\end{itemize}

\begin{Shaded}
\begin{Highlighting}[]
\NormalTok{peak_data_aligned <-}\StringTok{ }\KeywordTok{align_chromatograms}\NormalTok{(}\DataTypeTok{data =} \NormalTok{gc_peak_data, }\CommentTok{# input data}
    \DataTypeTok{conc_col_name =} \StringTok{"area"}\NormalTok{, }\CommentTok{# peak abundance variable}
    \DataTypeTok{rt_col_name =} \StringTok{"time"}\NormalTok{, }\CommentTok{# retention time }
    \DataTypeTok{rt_cutoff_low =} \DecValTok{5}\NormalTok{, }\CommentTok{# cut peaks with retention times below 5 Minutes}
    \DataTypeTok{rt_cutoff_high =} \DecValTok{45}\NormalTok{, }\CommentTok{# cut peaks with retention times above 45 Minutes}
    \DataTypeTok{reference =} \StringTok{"M3"}\NormalTok{, }\CommentTok{# name of reference }
    \DataTypeTok{max_linear_shift =} \FloatTok{0.05}\NormalTok{, }\CommentTok{# maximum linear shift of chromatograms}
    \DataTypeTok{max_diff_peak2mean =} \FloatTok{0.03}\NormalTok{, }\CommentTok{# maximum distance of a peak to the mean}
    \DataTypeTok{min_diff_peak2peak =} \FloatTok{0.03}\NormalTok{, }\CommentTok{# maximum distance between the mean of two peaks}
    \DataTypeTok{blanks =} \OtherTok{NULL}\NormalTok{, }\CommentTok{# no blanks. Specify blanks by names (e.g. c("blank1", "blank2"))}
    \DataTypeTok{delete_single_peak =} \OtherTok{TRUE}\NormalTok{, }\CommentTok{# delete peaks that are present in just one sample }
    \DataTypeTok{write_output =} \OtherTok{NULL}\NormalTok{) }\CommentTok{# add c("time","area") to write data frames to .txt file}
\end{Highlighting}
\end{Shaded}

\begin{Shaded}
\begin{Highlighting}[]
\KeywordTok{data}\NormalTok{(}\StringTok{"aligned_peak_data"}\NormalTok{)}
\end{Highlighting}
\end{Shaded}

\section{Evaluating the quality of the
alignment}\label{evaluating-the-quality-of-the-alignment}

\begin{Shaded}
\begin{Highlighting}[]
\KeywordTok{library}\NormalTok{(ggplot2)}
\KeywordTok{library}\NormalTok{(gridExtra)}
\end{Highlighting}
\end{Shaded}

\begin{Shaded}
\begin{Highlighting}[]
\KeywordTok{gc_heatmap}\NormalTok{(aligned_peak_data,}\DataTypeTok{threshold =} \FloatTok{0.01}\NormalTok{, }\DataTypeTok{samples_subset =} \DecValTok{1}\NormalTok{:}\DecValTok{20}\NormalTok{, }\DataTypeTok{substance_subset =} \DecValTok{1}\NormalTok{:}\DecValTok{30}\NormalTok{, }\DataTypeTok{label_size =} \DecValTok{10}\NormalTok{, }\DataTypeTok{type =} \StringTok{"continous"}\NormalTok{) }\CommentTok{# By default a threshold of 0.05 is used to mark deviations}
\end{Highlighting}
\end{Shaded}

\section{Algorithm}\label{algorithm}

\section{Evaluation with empirical data and
simulations}\label{evaluation-with-empirical-data-and-simulations}

\subsection{Availability}\label{availability}

The latest version of \texttt{GCalignR} can be downloaded from GitHub.

\begin{Shaded}
\begin{Highlighting}[]
\KeywordTok{install.packages}\NormalTok{(}\StringTok{"devtools"}\NormalTok{)}
\NormalTok{devtools::}\KeywordTok{install_github}\NormalTok{(}\StringTok{"mastoffel/GCalignR"}\NormalTok{)}
\end{Highlighting}
\end{Shaded}

We welcome any contributions or feedback on the package.

\subsection{Data accessibility}\label{data-accessibility}

\subsection*{References}\label{references}
\addcontentsline{toc}{subsection}{References}

\hypertarget{refs}{}
\hypertarget{ref-Allaire.2016}{}
Allaire, J. J., Joe Cheng, Yihui Xie, Jonathan McPherson, Winston Chang,
Jeff Allen, Hadley Wickham, Aron Atkins, and Rob Hyndman. 2016.
``Rmarkdown: Dynamic Documents for R.''
\url{https://CRAN.R-project.org/package=rmarkdown}.

\hypertarget{ref-Caspers.2011}{}
Caspers, Barbara A., Frank C. Schroeder, Stephan Franke, and Christian
C. Voigt. 2011. ``Scents of Adolescence: The Maturation of the Olfactory
Phenotype in a Free-Ranging Mammal.'' \emph{PloS One} 6 (6): e21162.

\hypertarget{ref-Charpentier.2010}{}
Charpentier, Marie J.E., Jeremy Chase Crawford, Marylène Boulet, and
Christine M. Drea. 2010. ``Message `Scent': Lemurs Detect the Genetic
Relatedness and Quality of Conspecifics via Olfactory Cues.''
\emph{Animal Behaviour} 80 (1): 101--8.
doi:\href{https://doi.org/10.1016/j.anbehav.2010.04.005}{10.1016/j.anbehav.2010.04.005}.

\hypertarget{ref-Dellicour.2013}{}
Dellicour, Simon, and Thomas Lecocq. 2013. ``GCALIGNER 1.0: An Alignment
Program to Compute a Multiple Sample Comparison Data Matrix from Large
Eco-Chemical Datasets Obtained by Gc.'' \emph{Journal of Separation
Science} 36 (19): 3206--9.
doi:\href{https://doi.org/10.1002/jssc.201300388}{10.1002/jssc.201300388}.

\hypertarget{ref-Gilad.2016}{}
Gilad, Oranit, Ronald R. Swaisgood, Megan A. Owen, and Xiaoping Zhou.
2016. ``Giant Pandas Use Odor Cues to Discriminate Kin from Nonkin.''
\emph{Current Zoology} 62 (4): 333--36.
doi:\href{https://doi.org/10.1093/cz/zow025}{10.1093/cz/zow025}.

\hypertarget{ref-Harris.2012}{}
Harris, Rachel L., Noel W. Davies, and Stewart C. Nicol. 2012.
``Chemical Composition of Odorous Secretions in the Tasmanian
Short-Beaked Echidna (Tachyglossus Aculeatus Setosus).'' \emph{Chemical
Senses} 37 (9): 819--36.
doi:\href{https://doi.org/10.1093/chemse/bjs066}{10.1093/chemse/bjs066}.

\hypertarget{ref-Jiang.2013}{}
Jiang, Wei, Zhi-Min Zhang, YongHuan Yun, De-Jian Zhan, Yi-Bao Zheng,
Yi-Zeng Liang, Zhen Yu Yang, and Ling Yu. 2013. ``Comparisons of Five
Algorithms for Chromatogram Alignment.'' \emph{Chromatographia} 76
(17-18): 1067--78.
doi:\href{https://doi.org/10.1007/s10337-013-2513-8}{10.1007/s10337-013-2513-8}.

\hypertarget{ref-Koh.2010}{}
Koh, Yueting, Kishore Kumar Pasikanti, Chun Wei Yap, and Eric Chun Yong
Chan. 2010. ``Comparative Evaluation of Software for Retention Time
Alignment of Gas Chromatography/Time-of-Flight Mass Spectrometry-Based
Metabonomic Data.'' \emph{Journal of Chromatography. A} 1217 (52):
8308--16.
doi:\href{https://doi.org/10.1016/j.chroma.2010.10.101}{10.1016/j.chroma.2010.10.101}.

\hypertarget{ref-Krause.2012}{}
Krause, E. Tobias, Oliver Krüger, Philip Kohlmeier, and Barbara A.
Caspers. 2012. ``Olfactory Kin Recognition in a Songbird.''
\emph{Biology Letters} 8 (3): 327--29.

\hypertarget{ref-Luedemann.2008}{}
Luedemann, Alexander, Katrin Strassburg, Alexander Erban, and Joachim
Kopka. 2008. ``TagFinder for the Quantitative Analysis of Gas
Chromatography--mass Spectrometry (Gc-Ms)-Based Metabolite Profiling
Experiments.'' \emph{Bioinformatics (Oxford, England)} 24 (5): 732--37.
doi:\href{https://doi.org/10.1093/bioinformatics/btn023}{10.1093/bioinformatics/btn023}.

\hypertarget{ref-Martin.2008}{}
Martin, Stephen J., Heikki Helanterä, and Falko P. Drijfhout. 2008.
``Evolution of Species-Specific Cuticular Hydrocarbon Patterns in
Formica Ants.'' \emph{Biological Journal of the Linnean Society} 95 (1):
131--40.
doi:\href{https://doi.org/10.1111/j.1095-8312.2008.01038.x}{10.1111/j.1095-8312.2008.01038.x}.

\hypertarget{ref-McNair.2011}{}
McNair, Harold M., and James M. Miller. 2011. \emph{Basic Gas
Chromatography}. John Wiley \& Sons.

\hypertarget{ref-Niu.2014}{}
Niu, Weihuan, Elisa Knight, Qingyou Xia, and Brian D. McGarvey. 2014.
``Comparative Evaluation of Eight Software Programs for Alignment of Gas
Chromatography-Mass Spectrometry Chromatograms in Metabolomics
Experiments.'' \emph{Journal of Chromatography. A} 1374: 199--206.
doi:\href{https://doi.org/10.1016/j.chroma.2014.11.005}{10.1016/j.chroma.2014.11.005}.

\hypertarget{ref-Oksanen.2016}{}
Oksanen, Jari, F. Guillaume Blanchet, Michael Friendly, Roeland Kindt,
Pierre Legendre, Dan McGlinn, Peter R. Minchin, et al. 2016. ``Vegan:
Community Ecology Package.''
\url{https://CRAN.R-project.org/package=vegan}.

\hypertarget{ref-Peng.2011}{}
Peng, Roger D. 2011. ``Reproducible Research in Computational Science.''
\emph{Science (New York, N.Y.)} 334 (6060): 1226--7.
doi:\href{https://doi.org/10.1126/science.1213847}{10.1126/science.1213847}.

\hypertarget{ref-Penn.2002}{}
Penn, Dustin J. 2002. ``The Scent of Genetic Compatibility: Sexual
Selection and the Major Histocompatibility Complex.'' \emph{Ethology}
108 (1): 1--21.
doi:\href{https://doi.org/10.1046/j.1439-0310.2002.00768.x}{10.1046/j.1439-0310.2002.00768.x}.

\hypertarget{ref-Pierce.2005}{}
Pierce, Karisa M., Janiece L. Hope, Kevin J. Johnson, Bob W. Wright, and
Robert E. Synovec. 2005. ``Classification of Gasoline Data Obtained by
Gas Chromatography Using a Piecewise Alignment Algorithm Combined with
Feature Selection and Principal Component Analysis.'' \emph{Journal of
Chromatography A} 1096 (1): 101--10.

\hypertarget{ref-Porter.1981}{}
Porter, R., and J. Moore. 1981. ``Human Kin Recognition by Olfactory
Cues☆.'' \emph{Physiology \& Behavior} 27 (3): 493--95.
doi:\href{https://doi.org/10.1016/0031-9384(81)90337-1}{10.1016/0031-9384(81)90337-1}.

\hypertarget{ref-RCoreTeam.2016}{}
R Core Team. 2016. ``R: A Language and Environment for Statistical
Computing.'' Vienna, Austria: R Foundation for Statistical Computing.
\url{https://www.R-project.org/}.

\hypertarget{ref-Robinson.2007}{}
Robinson, Mark D., David P. de Souza, Woon W. Keen, Eleanor C. Saunders,
Malcolm J. McConville, Terence P. Speed, and Vladimir A. Likić. 2007.
``A Dynamic Programming Approach for the Alignment of Signal Peaks in
Multiple Gas Chromatography-Mass Spectrometry Experiments.'' \emph{BMC
Bioinformatics} 8 (1): 419.

\hypertarget{ref-Smith.2006}{}
Smith, Colin A., Elizabeth J. Want, Grace O'Maille, Ruben Abagyan, and
Gary Siuzdak. 2006. ``XCMS: Processing Mass Spectrometry Data for
Metabolite Profiling Using Nonlinear Peak Alignment, Matching, and
Identification.'' \emph{Analytical Chemistry} 78 (3): 779--87.

\hypertarget{ref-Smith.2015}{}
Smith, Rob, Dan Ventura, and John T. Prince. 2015. ``LC-Ms Alignment in
Theory and Practice: A Comprehensive Algorithmic Review.''
\emph{Briefings in Bioinformatics} 16 (1): 104--17.
doi:\href{https://doi.org/10.1093/bib/bbt080}{10.1093/bib/bbt080}.

\hypertarget{ref-Stoffel.2015}{}
Stoffel, Martin A., Barbara A. Caspers, Jaume Forcada, Athina
Giannakara, Markus Baier, Luke Eberhart-Phillips, Caroline Müller, and
Joseph I. Hoffman. 2015. ``Chemical Fingerprints Encode
Mother--offspring Similarity, Colony Membership, Relatedness, and
Genetic Quality in Fur Seals.'' \emph{Proceedings of the National
Academy of Sciences} 112 (36): E5005--E5012.

\hypertarget{ref-Wyatt.2014}{}
Wyatt, Tristram D. 2014. \emph{Pheromones and Animal Behavior: Chemical
Signals and Signatures}. Cambridge University Press.

\hypertarget{ref-Zhang.2012}{}
Zhang, Zhi-Min, Yi-Zeng Liang, Hong-Mei Lu, Bin-Bin Tan, Xiao-Na Xu, and
Miguel Ferro. 2012. ``Multiscale Peak Alignment for Chromatographic
Datasets.'' \emph{Journal of Chromatography. A} 1223: 93--106.
doi:\href{https://doi.org/10.1016/j.chroma.2011.12.047}{10.1016/j.chroma.2011.12.047}.


\end{document}
