% !TeX root = RJwrapper.tex
\title{GCalignR. An R package for aligning Gas-Chromatography data}
\author{by Meinolf Ottensmann, Martin A. Stoffel, Joseph I. Hoffman}

\maketitle

\abstract{%
This is just a placeholder for 150 words in the abstract This is just a
placeholder for 150 words in the abstract This is just a placeholder for
150 words in the abstract This is just a placeholder for 150 words in
the abstract This is just a placeholder for 150 words in the abstract
This is just a placeholder for 150 words in the abstract This is just a
placeholder for 150 words in the abstract This is just a placeholder for
150 words in the abstract This is just a placeholder for 150 words in
the abstract This is just a placeholder for 150 words in the abstract
This is just a placeholder for 150 words in the abstract This is just a
placeholder for 150 words in the abstract This is just a placeholder for
150 words in the abstract
}

\section{Introduction}

Chemical cues are arguably the most common mode of communication among
animals \citep{Wyatt.2014}. By exploring broad patterns in complex
chemical signatures, researchers are enabled to make inferences on
kinship \citep{Krause.2012, Stoffel.2015}, genetic diversity
\citep{Charpentier.2010, Leclaire.2012}, sexual maturation
\citep{Caspers.2011} or species discrimination
\citep{Meulemeester.2011}. The characterisation of the involved
chemicals is accomplished using gas-chromatography, a fast
high-throughput method that unravels the composition of complex samples
by quantifying peaks \citep{McNair.2011}, while the additional
implementation of mass-spectrometry (GC-MS) allows to identify specific
substances, but not necessarily all \citep{Caspers.2011}. \par
However, before similarity patterns can be analysed, it is essential to
align compounds among samples, thereby accounting for drifts in the
retention times of peaks caused by subtle, random and often unavoidable
variations of the chromatography machine parameters \citep{Pierce.2005}.
Many studies rely on a manual alignment, but this approach bears three
severe drawbacks: (1) In large scale studies this task becomes
increasingly time consuming task and is impracticable. (2) Humans are
prone to detect patterns in noise which is why the researcher may bias
the alignment due to subjective experience and expectations. (3) The
data analytic pipeline from the raw gas-chromatography data to the
results of the statistical analysis is not reproducible. Several
alignment algorithms have been proposed to overcome these issues, but
these focus nearly exclusively on GC-MS data
\citep{Pierce.2005, Robinson.2007,Jiang.2013} and only some a easily
accessible as web-based tools \citep{Hoffmann.2009, Wang.2010} or
independent software \citep{Dellicour.2013}. \par
Here, we introduce \pkg{GCalignR}, a package that implements a simple
and fast algorithm to align peaks from GC data and evaluate the
resulting alignment using two data sets. \pkg{GCalignR} was specifically
developed as a tool for pre-processing GC data from animal skin and
preen glands prior to subsequent statistical analysis. In brief, the
algorithm consists of two main steps: (1) Systematic shifts of
chromatograms are corrected by applying appropriate linear shifts to
whole chromatograms based on a single reference. (2) Retention times of
individual peaks are step-wise grouped together with homologous peaks of
other samples and aligned within the same row in a retention time matrix
. The outcome of this grouping procedures can be altered by specifying
three parameters that are described in detail below. Among several
optional processing steps, the package allows to remove peaks belonging
to contaminations, which are identified due to their presence in control
samples. Furthermore, we demonstrate the easy integration of the
R-package
\href{https://cran.r-project.org/web/packages/vegan/index.html}{\CRANpkg{vegan}}
\citep{Oksanen.2016} into a solid workflow for multivariate analyses
starting with the raw chemical data, which can be fully integrated into
\strong{RMarkdown} documents \citep{Allaire.2016} to fulfil good
standards of reproducibility \citep{Peng.2011}.

\section{The Package}

\pkg{GCalignR} consists of functions that allow the alignment of peaks
from GC and GC-MS data based on retention times. The main aim of the
package is to provide a simple tool that guides the user towards the
alignment of large data sets prior to hypothesis-testing of the
multivariate data \citep{Anderson.2001}. We summarise the underlying
algorithm and workflow (figure \ref{figure:workflow}) below and refer to
the vignette that can be assessed via
\code{browseVignettes('GCalignR')}.

\begin{figure}[htbp]
  \centering
  \includegraphics[width=13cm]{figures/workflow}
  \caption{\pkg{GCalignR} workflow. In addition to the alignment of substances across samples, the package provides functions for checking and inspecting the data. The aligned data is ready to use for analyses in conjunction with other packages. Each function is explained within the text.}
  \label{figure:workflow}
\end{figure}

\subsection{Example datasets}

For demonstration purposes \pkg{GCalignR} includes data of chemical
signatures that were obtained by sampling the skin of 82 Antarctic fur
seals \textit{Arctocephalus gazella}. It was previously shown that these
signatures encode the membership to a breeding colony
\cite{Stoffel.2015}. These data are available as a \emph{List} with
individual samples included as a \emph{data.frame}. Two variables are
available that represent the required retention time (``time'') and peak
abundance (``area'').

\begin{Schunk}
\begin{Sinput}
library(GCalignR)
# Seal scent data
data("peak_data") 
# Data is organized in one list of data.frames
str(peak_data[1:2]) 
\end{Sinput}
\begin{Soutput}
#> List of 2
#>  $ C3:'data.frame':  217 obs. of  2 variables:
#>   ..$ time: num [1:217] 4.53 4.55 4.62 4.68 4.71 4.79 4.83 4.87 5.01 5.14 ...
#>   ..$ area: num [1:217] 3331224 1462381 4834211 7754401 1267617 ...
#>  $ C2:'data.frame':  217 obs. of  2 variables:
#>   ..$ time: num [1:217] 4.52 4.55 4.57 4.67 4.69 4.73 4.75 4.8 4.83 4.85 ...
#>   ..$ area: num [1:217] 2695110 5926253 10406833 6805905 1672849 ...
\end{Soutput}
\end{Schunk}

The second data set is comprised of three bumble bee species
\textit{Bombus bimaculatus}, \textit{B. ephippiatus} and
\textit{B. flavifrons} where signatures have been obtained from cephalic
labial gland secretions of 24, 20 and 11 individuals respectively
\citep{Dellicour.2013}. Cephalic secretions haven been demonstrated to
aid in species identification \citep{Meulemeester.2011}. These data are
available in a text file (figure \ref{figure:text}) in the format of
typical output from proprietary peak detection software (e.g.~XCalibur,
Thermofischer Scientific or Labsolution, Shimadzu). Here, the names of
samples and variables are given in the first two rows, while the peak
data is included from below with single matrices concatenated
horizontally.

\begin{figure}[htbp]
  \centering
  \includegraphics[width=8cm]{figures/text}
  \caption{Text files are the standard input of \pkg{GCalignR}}
  \label{figure:text}
\end{figure}

\subsection{Checking the input}

The package provides the function \strong{check{\_}input} that enables
the user to check the input for typical formatting errors and incomplete
data. We encourage to use unique names for samples that should consist
only of letters, numbers and underscores. If the data fails the test,
indicative warnings are returned that guide in correcting these.
Optionally, the argument \textit{list{\_}peaks} allows to create a
barplot of the initial peak distribution to get an overview of the data:

\begin{Schunk}
\begin{Sinput}
check_input(peak_data)
\end{Sinput}
\begin{Soutput}
#> All checks passed!
#> Ready for processing with align_chromatograms
\end{Soutput}
\begin{Sinput}
#> All checks passed!
#> Ready for processing with align_chromatograms
\end{Sinput}
\end{Schunk}

\section{Aligning substances among samples}

The alignment procedure is divided into five steps (figure
\ref{figure:algorithm}). All steps are executed by the main function
\textit{align{\_}chromatograms} and will be explained in in the next
sections. \subsubsection{Linear adjustments of chromatograms} At first,
chromatograms are linearly shifted with respect to a reference to
account for systematic shifts in retention times among homologous
chemicals shared by samples. Therefore, small linear adjustments are
applied to the entire set of peaks in a chromatogram (figure
\ref{figure:algorithm} A), such that the number of peaks that are shared
at a threshold of two decimals (i.e.~0.6 seconds) is maximised.The
parameter \textit{max{\_}linear{\_}shift} defines the maximum of linear
shifts that are considered by the program. \newline
Note: This method relies on the occurrence of substances that are shared
among most substances to produce efficient adjustments. If those are
absent, it is unlikely to find a suitable shift and chromatograms remain
untransformed. \par
A reference may be selected automatically by searching for the sample
with the highest average similarity to all other samples based on the
number of shared peaks prior to alignment. Alternatively, a chromatogram
may be included that contains peaks of an internal standard which peaks
are \textit{a-priori} known to occur in all samples. In this case, the
sample should be named \code{"reference"} and will be removed after the
alignment was conducted.

\newpage

\begin{figure}[htbp]
  \centering
  \includegraphics[width=13cm]{figures/algorithm}
  \caption{Overview of the algorithm performed by GCalignR. Row of matrices correspond to substances, columns are samples. Zeros indicate absence of peaks and are ignored in calculations. \strong{A}. Chromatograms are linearly shifted with respect to a reference (S2). Strong{B}. From left to right the first four steps from the input matrix to the final alignment are shown. Peaks are aligned row by row. Initially, always the second sample is compared to the first. Then the next sample is compared to all samples in previous columns until the last column is reached. Coloured cells represent conflicting retention times using \textit{max{\_}peak2mean = 0.02}. \strong{C}. After all peaks have been aligned, rows are merged depending on \textit{min{\_}peak2peak}, which defines the minimum difference that is expected between substances. If merging does not result in the loss of any data, rows are merged. \strong{D}. If specified, all peaks found in one or more blanks are removed as well as the blank itself. \strong{E}. Unique peaks which are present in only a single individual are not of interest for similarity analyses and can be removed as well.}
  \label{figure:algorithm}
\end{figure}

\subsubsection{Piece-wise alignment of substances}

The core of the alignment procedure is based on clustering of individual
peaks among samples. This is performed by examining retention times
within single rows, where samples are compared consecutively with all
previous samples starting with the second column (figure
\ref{figure:algorithm} B):\par
If

\begin{equation}
rt_{m} > \left(\frac{\sum_{i=1}^{m-1}rt_{i}}{m-1}\right) + max{\_}peak2mean
\end{equation}

the examined peak is moved into the next row, whereas all previous
samples are moved \par
if

\begin{equation}
rt_{m} < \left(\frac{\sum_{i=1}^{m-1}rt_{i}}{m-1}\right) - max{\_}peak2mean
\end{equation}

with \textit{rt} = retention time; \textit{m} = current column and
\textit{max{\_}peak2mean} defining the maximal deviation of the mean
retention time. \newline By considering the mean retention time among
all previous samples the algorithm accounts for substance specific
variations, such that less variable retention times are treated more
stringent than chemicals exhibiting higher variability. Once the last
retention time of a row was evaluated the whole procedure is repeated
with the next row until the end of the retention time matrix was
reached. Afterwards, rows with similar mean retention times are assessed
for redundancy (figure \ref{figure:algorithm} C), which applies whenever
a merging does not cause any loss of any information (i.e.~no sample
exists that contains substances in both rows). The similarity threshold
is given by \textit{min{\_}peak2peak} defining the minimal difference
between peaks that is expected. \par In combination these two fairly
simple algorithms align compounds by considering compound-specific
variation in both steps.

\subsubsection{Normalisation}

Many multivariate analysis techniques, like those available in
\pkg{vegan}, require a data frame of independent variables as input
format. Moreover is generally advisable to normalise abundancies prior
to statistical analysis to correct for variations in the total
concentration of samples. This is utilised in \pkg{GCalignR´s} function
\code{normalise_peaks} which normalises peak abundancies by calculating
realtive abundancies for each sample.

\section{Workflow}

Here, we demonstrate the typical workflow using our seal data. This is
done using the function \code{align{\_}chromatograms}. A list of all
parameters and their description can be assessed from the documentation
in the helpfile by typing \code{?align{\_}chromatograms}:

\begin{Schunk}
\begin{Sinput}
seal_aligned <- align_chromatograms(data = peak_data,
                    conc_col_name = "area",
                    max_diff_peak2mean = 0.02,
                    min_diff_peak2peak = 0.05,
                    max_linear_shift = 0.05,
                    rt_col_name = "time",
                    delete_single_peak = T,
                    blanks = c("C2","C3"))
\end{Sinput}
\end{Schunk}

Now, we can inspect the results by retrieving summaries of the alignment
process. The printing method summarises the function call including
defaults that have not been explicitly specified during the function
call. We also get the relevant information to retrace every step in the
alignment:

\begin{Schunk}
\begin{Sinput}
print(seal_aligned)
\end{Sinput}
\begin{Soutput}
#>   Summary of Peak Alignment running align_chromatograms from package GCalignR
#>   Input: peak_data   Start:  2016-08-16 17:21:23     Finished:  2016-08-16 17:29:59 
#> 
#> Call:
#>   GCalignR::align_chromatograms(data=peak_data, conc_col_name=area,
#>   rt_col_name=time, rt_cutoff_low=8, reference=M29, max_linear_shift=0.05,
#>   max_diff_peak2mean=0.02, min_diff_peak2peak=0.03, blanks=(C2,
#>   C3), delete_single_peak=T, sep=\t, rt_cutoff_high=NULL, n_iter=1,
#>   merge_rare_peaks=FALSE)
#> 
#> Summary of scored substances:
#> 
#>     Peaks In_Blanks  Singular  Retained 
#>       416       119        70       227 
#> 
#>   In total 416 substances were identified among all samples. NA substances were
#>   present in blanks. The corresponding peaks as well as the blanks were removed
#>   from the data set. 70 substances were present in just one single sample and were
#>   removed. 227 substances are retained after all filtering steps.
#> 
#> Sample Overview  The following 84 Samples were aligned to the reference 'M29':
#>   M2, M3, M4, M5, M6, M7, M8, M9, M10, M12, M14, M15, M16, M17, M18, M19, M20,
#>   M21, M23, M24, M25, M26, M27, M28, M29, M30, M31, M33, M35, M36, M37, M38, M39,
#>   M40, M41, M43, M44, M45, M46, M47, M48, P2, P3, P4, P5, P6, P7, P8, P9, P10,
#>   P12, P14, P15, P16, P17, P18, P19, P20, P21, P23, P24, P25, P26, P27, P28, P29,
#>   P30, P31, P33, P35, P36, P37, P38, P39, P40, P41, P43, P44, P45, P46, P47, P48
#> 
#> For further details:
#>   Type 'gc_heatmap(seal_aligned)' to retrieve a heatmap for the alignment accuracy
#>   Type 'plot(seal_aligned)' to retrieve further diagnostic plots
\end{Soutput}
\end{Schunk}

The quality of an alignment will depend on sensible parameters that
facilitate the (i) correction of linear shifts that might fall in a
larger range with increasing sample size and (ii) and the variability of
retention times. Optimally, linear shifts do not exhaust the range given
by \code{max{\_}linear{\_}shift} completely, which would in turn
indicate that not all uncertainties haven been fully compensated for.
This can be assessed by some diagnostic plots:

\begin{Schunk}
\begin{Sinput}
plot(seal_aligned)
\end{Sinput}

\includegraphics{ottensmann-stoffel-hoffman_files/figure-latex/unnamed-chunk-6-1} \end{Schunk}

\bibliography{ottensmann-stoffel-hoffman}

\address{%
Meinolf Ottensmann\\
Department of Animal Behaviour\\
Bielefeld University\\ Morgenbreede 45\\ 33615 Bielefeld\\
}
\href{mailto:Meinolf.Ottensmann@web.de}{\nolinkurl{Meinolf.Ottensmann@web.de}}

\address{%
Martin A. Stoffel\\
Department of Animal Behaviour\\
Bielefeld University\\ Morgenbreede 45\\ 33615 Bielefeld\\
}
\href{mailto:Martin.Adam.Stoffel@gmail.com}{\nolinkurl{Martin.Adam.Stoffel@gmail.com}}

\address{%
Joseph I. Hoffman\\
Department of Animal Behaviour\\
Bielefeld University\\ Morgenbreede 45\\ 33615 Bielefeld\\
}
\href{mailto:j_i_hoffman@hotmail.com}{\nolinkurl{j\_i\_hoffman@hotmail.com}}

