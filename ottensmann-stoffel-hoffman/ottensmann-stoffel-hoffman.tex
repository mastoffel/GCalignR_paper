% !TeX root = RJwrapper.tex
\title{GCalignR. An R package for aligning Gas-Chromatography data}
\author{by Meinolf Ottensmann, Martin A. Stoffel, Joseph I. Hoffman}

\maketitle

\abstract{%
Chemical signals are among the most fundamental and oldest means of
animal communication. The desire to unravel broader patterns of chemical
communication in birds and mammals paved the way for two not entirely
new techniques, gas-chromatography and mass-spectrometry, in the fields
of ecology and evolution. Comparing chemical profiles or chromatograms
across many individuals yields some major obstacles as even the newest
GC machines have an inherent error when measuring the retention times of
chemical substances. Here we present GCalignR, an R package for the
alignment of chromatography peaks among samples prior to hypothesis
testing using multivariate statistics. GCalignR is specifically designed
to be used by non-chemists by providing easy to use functions to check
and align gas-chromatography data based on retention times. In addition,
the package implements heatmaps and other plots to evaluate and
potentially adjust the peak alignment. We hope that GCalignR will
provide a tool that fits into a common biologist's workflow in R and
that the package will contribute to the standardization and
reproducibility of studies on chemical communication.
}

\subsection{Introduction}\label{introduction}

Chemical cues are arguably the most common mode of communication among
animals \citep{Wyatt.2014}. Patterns in complex chemical signatures can
can yield information about kinship \citep{Krause.2012, Stoffel.2015},
genetic diversity \citep{Charpentier.2010, Leclaire.2012}, sexual
maturation \citep{Caspers.2011} or be used for species discrimination
\citep{Meulemeester.2011}. One of the most common instruments to
quantify the chemical composition of samples is gas-chromatography (GC),
a fast high-throughput method to detect individual chemicals and their
abundancies \citep{McNair.2011}, while the additional implementation of
mass-spectrometry (GC-MS) allows to identify specific substances
\citep{Caspers.2011}. \par
However, before similarity patterns across samples can be analysed, it
is essential to align compounds. The alignment of samples has to account
for pertubations in the retention times of peaks which are caused by
subtle, random and often unavoidable variations of the chromatography
machine parameters such as as changes of the ambient temperature, flow
rate of the carrier gas or column aging \citep{Scott.2003, Pierce.2005}.
Surprisingly, studies on mammalian or avian chemical communication often
rely on manual alignment rather than (semi-)automated algorithms, but
this approach bears three severe drawbacks: (1) For larger sample sizes,
this task becomes extremely time consuming and inefficient (2) The
researcher may bias the alignment due to subjective experience and
expectations. (3) The data analytical pipeline from the raw
gas-chromatography data to the results of the statistical analysis is
not reproducible. Several alignment algorithms have been proposed to
overcome these issues, but these focus nearly exclusively on GC-MS data
\citep{Pierce.2005, Robinson.2007,Jiang.2013} and only some are
implemented in web-based tools \citep{Hoffmann.2009, Wang.2010} or are
available as independent software \citep{Dellicour.2013}. \par
Here, we introduce \texttt{GCalignR}, an R package that implements a
simple algorithm to align peaks purely based on retention time data
obtained by GC and provides sophisticated visualisations for the
evaluation of the alignment quality. \texttt{GCalignR} was specifically
developed as a tool for pre-processing GC data from animal skin and
avian preen glands prior to subsequent statistical analysis. In brief,
the algorithm consists of two main steps: (1) Systematic shifts of
chromatograms are corrected by applying appropriate linear shifts to
whole chromatograms based on a single reference. (2) Retention times of
individual peaks are grouped iteratively together with homologous peaks
of other samples and aligned within the same row in a retention time
matrix. The quality of this grouping procedure can be adjusted by three
parameters that are described in detail below. Among several optional
processing steps, the package allows to remove peaks that represent
contaminations, which are identified due to their presence in negative
control samples, henceforth called blanks. For an easy interpretation of
the quality of an alignment we implemented several diagnostic plots that
allow to access the aligned data visually. Furthermore, we demonstrate a
complete workflow from chemical raw data to multivariate analyses with
the popular and widely used
\href{https://cran.r-project.org/web/packages/vegan/index.html}{\CRANpkg{vegan}}
\citep{Oksanen.2016} package. This enables researches to make their work
reproducible by others by an easy integration of all computational steps
into \strong{RMarkdown} documents \citep{Allaire.2016} for instance,
thereby fulfilling good standards of reproducibility \citep{Peng.2011}.

\subsection{The Package}\label{the-package}

\texttt{GCalignR} contains functions to align peaks from GC and GC-MS
data based on retention times and evaluate the respective alignments.
The main aim of the package is to provide a simple tool that guides the
user through the alignment of large data sets prior to the statistical
analysis of multivariate chemical data. An typical workflow for the
analysis of chemical signatures including \texttt{GCalignR} is shown in
figure \ref{figure:workflow} and described below. The package vignette
provides a detailed description of all functions and their arguments and
can be assessed via \code{browseVignettes('GCalignR')} once the package
was installed.

\begin{figure}[htbp]
\centering
\includegraphics[width=13cm]{figures/workflow}
\caption{\pkg{GCalignR} workflow. In addition to the alignment of substances across samples, the package provides functions for checking and inspecting the data. The aligned data is ready to use for analyses in conjunction with other packages. Each function is explained within the text.}
\label{figure:workflow}
\end{figure}

\subsubsection{Example dataset}\label{example-dataset}

For demonstration purposes \texttt{GCalignR} includes data of skin
chemicals from 82 Antarctic fur seals \textit{Arctocephalus gazella}. It
was previously shown that these signatures encode the membership to a
breeding colony \cite{Stoffel.2015}. These data are available in a
single text file, the standard input format of \texttt{GCalignR}, that
is distributed with the package. The first two lines contain the names
of all samples and variables respectively. From the third row onwards,
data of all samples is included, whereby data frames are concatenated
horizontally.

\begin{Schunk}
\begin{Sinput}
## Load the package
library(GCalignR)
## Path to the dataset
fpath <- system.file("extdata", "peak_data.txt", package = "GCalignR")
\end{Sinput}
\end{Schunk}

\begin{Schunk}
\begin{Sinput}
## Open the file in an external editor
file.show(fpath)
\end{Sinput}
\end{Schunk}

\subsection{Alignment of Gas-Chromatography peaks among
samples}\label{alignment-of-gas-chromatography-peaks-among-samples}

The alignment procedure is divided into five steps (figure
\ref{figure:algorithm}). All steps are executed within the main function
\texttt{align\_chromatograms} and will be explained in in the next
sections.

\subsubsection{(1) Linear adjustments of
chromatograms}\label{linear-adjustments-of-chromatograms}

At first, all peaks within a chromatogram are shifted with respect to a
reference chromatogram to account for systematic shifts in retention
times among homologous chemicals shared by samples (figure
\ref{figure:algorithm} A). This is done for all samples in relation to
the reference sample such that the number of shared peaks is maximised.
The parameter \emph{max\_linear\_shift} defines the maximum temporal
range of linear shifts that are considered by the program. \newline
Note: This method relies on the occurrence of substances that are shared
among most substances to produce efficient adjustments. If those are
absent, it is unlikely to find a suitable shift and chromatograms remain
untransformed after this step. \par
A reference is selected automatically by searching for the sample with
the highest average similarity to all other samples based on the number
of shared peaks prior to alignment. Alternatively, a chromatogram can be
included that contains peaks of an internal standard which peaks are
\emph{a-priori} known to occur in all samples. In this case, the sample
may be named \texttt{reference} and will be removed after the alignment
was conducted.

\begin{figure}[htbp]
\centering
\includegraphics[width=13cm]{figures/algorithm_representation}
\caption{Overview of the algorithm performed by GCalignR using. Rows of matrices correspond to substances, columns are samples. Zeros indicate absence of peaks and are therefore not considered in computations. \textbf{A}. Chromatograms are linearly shifted with respect to a reference (here S2). \textbf{B}. From left to right the first four steps from the input matrix to the final alignment are shown. Peaks are aligned row by row. Initially, always the second sample is compared to the first. Then the next sample is compared to all samples in previous columns until the last column is reached. \textbf{C}. Coloured cells represent conflicting retention times of assembled rows that show a smaller difference than specified. If merging does not result in the loss of any data, rows are merged. \textbf{D}. If specified, all peaks found in one or more blanks (negative controls) are removed as well as the blank itself. \textbf{E}. Unique peaks present in a single individual are not of interest for similarity analyses and can be removed as well.}
\label{figure:algorithm}
\end{figure}

\subsubsection{(2) Peak alignment}\label{peak-alignment}

The core of the alignment procedure is based on clustering of individual
peaks across samples. This is performed by examining retention times
within single rows, where samples are compared consecutively with all
previous samples starting with the second column (figure
\ref{figure:algorithm} B):

\begin{equation}
rt_{m} > \left(\frac{\sum_{i=1}^{m-1}rt_{i}}{m-1}\right) + max\textunderscore diff\textunderscore peak2mean
\end{equation}

If the examined peak is moved into the next row, whereas all previous
samples are moved

\begin{equation}
rt_{m} < \left(\frac{\sum_{i=1}^{m-1}rt_{i}}{m-1}\right) - max\textunderscore diff\textunderscore peak2mean
\end{equation}

with \emph{rt} = retention time; \emph{m} = current column and
\emph{max\_diff\_peak2mean} defining the maximal deviation of the mean
retention time.

By considering the mean retention time among all previous samples the
algorithm accounts for substance specific variations, such that less
variable retention times are treated more stringent than chemicals
exhibiting higher variability. Once the last retention time of a row was
evaluated the whole procedure is repeated with the next row until the
end of the retention time matrix was reached.

\subsubsection{(3) Merging}\label{merging}

Sometimes, a single substance has been split up into two different rows.
However, the emerging pattern can be very clear, as a part of the
samples will have the substance in a given row, but no substance in the
adjacent row and vice versa for another set of samples. Knowing this
pattern, rows will be merged when this does not cause any loss of
information (i.e.~no sample exists that contains substances in both
rows).(figure \ref{figure:algorithm} C). Again, the user can change the
threshold for the minimal difference in the retention time between two
mergeable peaks with \emph{min\_diff\_peak2peak}. \par 

\subsubsection{(4) Post processing}\label{post-processing}

After aligning peaks the package offers several optional post processing
steps that allow to clean-up the data.

\subsubsection{Removing contaminations}\label{removing-contaminations}

Among other sources, residues of unwanted chemical substances in the gas
chromatography column or within reagents used in the laboratory have the
potential to contaminate chemical samples. To get rid of these
substances it is generally advised to include control samples. Within
\texttt{align\_chromatograms} those controls can be included in the data
set in the same way as a normal sample. By specifying the name of one or
more negative control samples as parameter \texttt{blanks} all
substances present in the control samples are removed from the dataset.

\subsubsection{Removing single peaks}\label{removing-single-peaks}

Sometimes, substances occur purely in a single sample. For comparative
approaches that calculate similarity matrices these substances are often
not informative and can be removed from the data. \texttt{GCalignR}
allows to do so by setting the \texttt{delete\_single\_peak} argument to
\texttt{TRUE}.

\subsubsection{Normalisation}\label{normalisation}

Many multivariate analysis techniques, like those available in
\pkg{vegan}, require a \texttt{data\ frame} of independent variables as
input format. Moreover it is generally advisable to normalise substance
abundancies prior to statistical analysis to correct for variations in
the total concentration of samples. This can be done in
\texttt{GCalignR} with the function \texttt{normalise\_peaks} which
calculates relative abundancies within each sample. Even if the dataset
already included the relative area for instance, a normalisation has to
be repeated after any filtering of peaks was conducted that altered the
total concentration.

\subsection{Workflow}\label{workflow}

Here, we demonstrate a typical workflow in \texttt{GCalignR} using our
seal data. All alignment steps that have been described above are
implemented within the function \texttt{align\_chromatograms}. A list of
all parameters and their description can be assessed from the
documentation in the helpfile by typing \texttt{?align\_chromatograms}.
As it is outlined in figure \ref{figure:workflow}, the package provides
the function \texttt{check\_input} to test the input file for typical
formatting errors and incomplete data. We encourage to use unique names
for samples that consist only of letters, numbers and underscores. If
the data fails the test, indicative warnings are returned which guide in
correcting those errors. This function is executed internally prior to
any alignment, hence the data needs to pass the checks.

\begin{Schunk}
\begin{Sinput}
check_input(fpath)
\end{Sinput}
\begin{Soutput}
#> All checks passed!
\end{Soutput}
\end{Schunk}

\begin{Schunk}
\begin{Sinput}
aligned_peak_data <- align_chromatograms(data = peak_data,
        rt_col_name = "time",
        max_diff_peak2mean = 0.02,
        min_diff_peak2peak = 0.08,
        max_linear_shift = 0.05,
        delete_single_peak = TRUE,
        blanks = c("C2","C3"), 
        write_output = NULL) # change to generate text files
\end{Sinput}
\end{Schunk}

Now, we can inspect the results by retrieving summaries of the alignment
process. The printing method summarises the function call including
defaults that have not been explicitly specified during the function
call. We also get the relevant information to retrace every step in the
alignment:

\begin{Schunk}
\begin{Sinput}
print(aligned_peak_data)
\end{Sinput}
\begin{Soutput}
#> Summary of Peak Alignment running align_chromatograms
#> Input: peak_data
#> Start:  2017-02-01 18:04:11  Finished:  2017-02-01 18:41:11 
#> 
#> Call:
#>   GCalignR::align_chromatograms(data=peak_data, rt_col_name=time,
#>   max_linear_shift=0.05, blanks=(C2, C3), sep=\t, rt_cutoff_low=NULL,
#>   rt_cutoff_high=NULL, reference=NULL, max_diff_peak2mean=0.02,
#>   min_diff_peak2peak=0.08, delete_single_peak=FALSE)
#> 
#> Summary of scored substances:
#>    total   blanks retained 
#>      490      171      319 
#> 
#> In total 490 substances were identified among all samples. 171 substances were
#>   present in blanks. The corresponding peaks as well as the blanks were removed
#>   from the data set. 319 substances are retained after all filtering steps.
#> 
#> Sample overview:
#>   The following 84 samples were aligned to the reference 'P31':
#>   M2, M3, M4, M5, M6, M7, M8, M9, M10, M12, M14, M15, M16, M17, M18, M19, M20,
#>   M21, M23, M24, M25, M26, M27, M28, M29, M30, M31, M33, M35, M36, M37, M38, M39,
#>   M40, M41, M43, M44, M45, M46, M47, M48, P2, P3, P4, P5, P6, P7, P8, P9, P10,
#>   P12, P14, P15, P16, P17, P18, P19, P20, P21, P23, P24, P25, P26, P27, P28, P29,
#>   P30, P31, P33, P35, P36, P37, P38, P39, P40, P41, P43, P44, P45, P46, P47, P48
#> 
#> For further details type...
#>   'gc_heatmap(aligned_peak_data)' to retrieve heatmaps
#>   'plot(aligned_peak_data)' to retrieve further diagnostic plots
\end{Soutput}
\end{Schunk}

The quality of an alignment will depend on sensible parameters that
facilitate the (\emph{i}) correction of linear shifts that might fall in
a larger range with increasing sample size and (\emph{ii}) and the
variability of retention times. Optimally, linear shifts do not exhaust
the range given by \texttt{max\_linear\_shift} completely, which would
in turn indicate that not all systematic deviations haven been fully
compensated. This can be assessed by four diagnostic plots that can be
created altogether as well as individually.

\begin{Schunk}
\begin{Sinput}
plot(aligned_peak_data)
\end{Sinput}
\begin{figure}

{\centering \includegraphics{ottensmann-stoffel-hoffman_files/figure-latex/unnamed-chunk-6-1} 

}

\caption[Diagnostic plots summarise aligned datasets]{Diagnostic plots summarise aligned datasets}\label{fig:unnamed-chunk-6}
\end{figure}
\end{Schunk}

The distribution of peak numbers before and after the alignment reveals
a noticeable reduction of peaks in the aligned dataset. These changes
can be explained by the removal of contaminations (i.e.~peaks present in
blanks) and the removal of single peaks. Type
\texttt{print(aligned\_peak\_data)} for details on both. The
distribution of shifts sizes used for linear transformations shows a
marginal linear trend across the chromatography run and will depend on
the position of samples relative to the reference while performing the
chromatography. Besides the pure number of peaks it is of major interest
to inspect the distribution of substances in the pool of samples and
access the variation in retention times. This can be investigated
simultaneously with a heatmap.

\begin{Schunk}
\begin{Sinput}
gc_heatmap(aligned_peak_data,type = "discrete",
           substance_subset = 1:25, samples_subset = 1:25)
\end{Sinput}
\begin{figure}

{\centering \includegraphics{ottensmann-stoffel-hoffman_files/figure-latex/unnamed-chunk-7-1} 

}

\caption[Heatmaps allow to inspect the distribution of substances across samples as well as the variability of their retention times]{Heatmaps allow to inspect the distribution of substances across samples as well as the variability of their retention times.}\label{fig:unnamed-chunk-7}
\end{figure}
\end{Schunk}

The heatmap indicates the presence of a certain substance within a
sample by a colour-filled box, whereas the absence is encoded by a white
box. Furthermore a colour-gradient
(\texttt{type\ =\ \textquotesingle{}discrete\textquotesingle{}}) is used
to indicate the deviation of each retention time from the mean value
among all other samples as a measure of variation. The function
\texttt{gc\_heatmap} offers different possibilities to inspect subsets
of samples and substances. The user can benefit the most by using
\texttt{gc\_heatmap} in a dynamic visualisation process that enables to
quickly get an overview of the whole datset as well as finer details of
interest. The pattern shown here does not indicate any obvious issues
with the aligned dataset. Hence, there is no need to adjust the aligning
parameters further and we can move on to analyse the dataset. Prior to
analysing pattern within the aligned data we normalise the peak area to
correct for difference in the total concentration among samples and
format the data to subsequent ordination approaches using \pkg{vegan}.

\begin{Schunk}
\begin{Sinput}
## normalise the peak area 
scent <- norm_peaks(data = aligned_peak_data,
                   rt_col_name = "time",
                   conc_col_name = "area",
                   out = "data.frame" )
\end{Sinput}
\end{Schunk}

\subsubsection{Visualising patterns by
ordination}\label{visualising-patterns-by-ordination}

\pkg{vegan} offers several methods for ordination approaches. We apply a
non-metric-multidimensional scaling (NMDS) using a Bray-Curtis
dissimilarity in order to investigate differences in chemical profiles
between two colonies. The package contains a data frame
\texttt{peak\_factors} that is comprised of three factors
(``colony'',``family'' and ``age'') for all samples that denote the
rownames. The visualisation is done using \pkg{ggplot2}.

\begin{Schunk}
\begin{Sinput}
## factors for each sample of the chemical dataset 
head(peak_factors) 
\end{Sinput}
\begin{Soutput}
#>     colony family age
#> M10    FWB     10   1
#> M12    FWB     12   1
#> M14    FWB     14   1
#> M15    SSB     15   1
#> M16    FWB     16   1
#> M17    FWB     17   1
\end{Soutput}
\begin{Sinput}
## both datasets have the same rownames and can be sorted accordingly
scent <- scent[match(row.names(peak_factors),row.names(scent)),] 
## standard log + 1 transformation
scent <- log(scent + 1)
\end{Sinput}
\end{Schunk}

\begin{Schunk}
\begin{Sinput}
## NMDS using Bray-Curtis dissimilarities
scent_nmds <- vegan::metaMDS(comm = scent,distance = "bray")
## get the x and y coordinates
scent_nmds <- as.data.frame(scent_nmds[["points"]]) 
## add the factor of interest
scent_nmds <- cbind(scent_nmds,colony = peak_factors[["colony"]]) 
\end{Sinput}
\end{Schunk}

\textbackslash{}begin\{Schunk\}

\begin{Sinput}
## ordiplot with ggplot2
library(ggplot2)
ggplot(data = scent_nmds,aes(MDS1,MDS2,color = colony)) +
    geom_point(size = 4) + 
    stat_ellipse(size = 2) + 
    labs(title = "", x = "MDS1", y = "MDS2") +  
    theme_void() + 
    theme(panel.background = element_rect(colour = "black", size = 2,fill = NA),
          aspect.ratio = 1)
\end{Sinput}

\textbackslash{}begin\{figure\}

\{\centering \includegraphics{ottensmann-stoffel-hoffman_files/figure-latex/unnamed-chunk-11-1}

\}

\textbackslash{}caption{[}A NMDS plot shows the similarity of
individuals within colonies{]}\{A NMDS plot shows the similarity of
individuals within colonies. Ellipses are drawn at the 95\% confidence
level.\}\label{fig:unnamed-chunk-11} \textbackslash{}end\{figure\}
\textbackslash{}end\{Schunk\}

The ordination plot shows a clear pattern that separates individuals by
the breeding colony. \pkg{vegan} offers furthermore permutational test
for multivariate analysis of variance (``permutational manova'',
\citep{Anderson.2001}) that support the observed pattern.

\begin{Schunk}
\begin{Sinput}
## Testing for a location effect
vegan::adonis(scent ~ peak_factors[["colony"]],permutations = 9999)
\end{Sinput}
\begin{Soutput}
#> 
#> Call:
#> vegan::adonis(formula = scent ~ peak_factors[["colony"]], permutations = 9999) 
#> 
#> Permutation: free
#> Number of permutations: 9999
#> 
#> Terms added sequentially (first to last)
#> 
#>                          Df SumsOfSqs MeanSqs F.Model     R2 Pr(>F)    
#> peak_factors[["colony"]]  1    2.5351 2.53514  11.492 0.1256  1e-04 ***
#> Residuals                80   17.6486 0.22061         0.8744           
#> Total                    81   20.1837                 1.0000           
#> ---
#> Signif. codes:  0 '***' 0.001 '**' 0.01 '*' 0.05 '.' 0.1 ' ' 1
\end{Soutput}
\begin{Sinput}
## Testing for a dispersion effect
anova(vegan::betadisper(vegan::vegdist(scent,method = "bray"),peak_factors[["colony"]]))
\end{Sinput}
\begin{Soutput}
#> Analysis of Variance Table
#> 
#> Response: Distances
#>           Df   Sum Sq   Mean Sq F value Pr(>F)
#> Groups     1 0.000347 0.0003474   0.095 0.7587
#> Residuals 80 0.292452 0.0036557
\end{Soutput}
\end{Schunk}

\subsection{Validation}\label{validation}

We analysed the performance of \texttt{GCalignR} using datasets of three
bumble bee species \emph{Bombus bimaculatus}, \emph{B. ephippiatus} and
\emph{B. flavifrons} where signatures have been obtained from cephalic
labial gland secretions of 24, 20 and 11 individuals respectively. These
data have been published as supplementary material by
\citet{Dellicour.2013}. Moreover, for a subset of peaks, substances have
been identified by GC-MS. We used all identified substances (\emph{B.
bimaculatus}, 717 peaks of 32 substances; \emph{B. ephippiatus}, 782 and
42; \emph{B. flavifrons} 457 and 44) to determine error rates for our
alignments. Hence, we calculated the error rate as the ratio between
incorrectly assigned retention times and the total number of retention
times:

\begin{equation}
Error = \left[\frac{N_{missaligned}}{N_{total}}\right] 
\end{equation}

whereby \emph{N} denotes the number of retention times and peaks where
treated as missaligned whenever they were not assigned to the row that
defines the mode of a given substance. By systematically changing the
two parameters \texttt{max\_diff\_peak2mean} and
\texttt{min\_diff\_peak2peak} we explored 100 parameter combinations to
demonstrate how parameter values affect the alignment accuracy. \par
Additionally, we simulated the effect of noise (i.e.~bad quality
chromatograms) on the error rate by addition or subtraction of 0.02 or
0.01 minutes to a random subset of peaks per sample. Code and
documentation are provided in a single PDF file written in Rmarkdown
(S1) together with the data (S2).

\subsubsection{Results}\label{results}

\begin{figure}[htbp]
\centering
\includegraphics[width=13cm]{figures/parameter_space}
\caption{Effects of alignment parameters on the error rate of three datasets where the identity of a subset of peaks was confirmed by GC-MS \citep{Dellicour.2013}. Each point shows the error in aligning substances for a combination of max\textunderscore diff\textunderscore peak2mean and min\textunderscore diff\textunderscore peak2peak.}
\label{figure:parameterspace}
\end{figure}

The parameter \texttt{min\_diff\_peak2peak} showed the strongest
influence on the error rates based on all three bumblebee datasets
(figure \ref{figure:parameterspace}). With values below 0.06 minutes the
error rate was modest for all datasets. The combination of
\texttt{min\_diff\_peak2peak\ =\ 0.11} and
\texttt{max\_diff\_peak2mean\ =\ 0.04} offered the lowest average error
rate of 3.24 \% (\emph{B. bimaculatus} = 2.79\%, \emph{B. ephippiatus} =
3.20\%, \emph{B. flavifrons} = 3.72\%).

\begin{figure}[htbp]
\centering
\includegraphics[width=13cm]{figures/noise_simulation}
\caption{Additional noise in peak retention times increases the error rate substantially. Therefore, optimal alignments require clearly resolved peaks that need to be extracted prior to using \pkg{GCalignR}}
\label{figure:noise}
\end{figure}

Adding additional noise (figure \href{figure:noise}) to the raw
retention times increased the error rate in the aligned data
substantially. Therefore, we emphasise the importance to check datasets
prior to alignment and elliminate uncertain peaks early on.

\subsection{Final remarks}\label{final-remarks}

\texttt{GcalignR} is primarily intended as a preprocessing tool in the
analysis of complex chemical signatures of animals where similarity
patterns are of interest rather than analysing the role of specific
compounds. Therefore, we prioritise an objective and fast alignment that
is not claimed to be freed of any errors, which are not possible to
avoid without additional sources of information such as GC-MS. It is in
the responsibility of the user to critically analysis the data before
and after alignment.

\subsection{Availability}\label{availability}

The current stable version requires at least R 3.2.5 and is available on
\texttt{CRAN}.

\begin{Schunk}
\begin{Sinput}
install.packages("GCalignR")
\end{Sinput}
\end{Schunk}

We aim to extend the functionalities of \pkg{GCalignR} in future and the
developmental version can be downloaded from GitHub within R using
\href{https://cran.r-project.org/web/packages/devtools/index.html}{\CRANpkg{devtools}}
\citep{Wickham.2016}.

\begin{Schunk}
\begin{Sinput}
devtools::install_github("mastoffel/GCalignR", build_vignettes = TRUE)
\end{Sinput}
\end{Schunk}

\subsection{Data accessability}\label{data-accessability}

The Fur seal dataset is included in this R package, the bumblebee
datasets are available as supplementary material S2 (see also the
Rmarkdown file S1).

\subsection{Acknowledgements}\label{acknowledgements}

\bibliography{ottensmann-stoffel-hoffman}

\address{%
Meinolf Ottensmann\\
Department of Animal Behaviour\\
Bielefeld University\\ Morgenbreede 45\\ 33615 Bielefeld\\
}
\href{mailto:meinolf.ottensmann@web.de}{\nolinkurl{meinolf.ottensmann@web.de}}

\address{%
Martin A. Stoffel\\
Department of Animal Behaviour\\
Bielefeld University\\ Morgenbreede 45\\ 33615 Bielefeld\\
}
\href{mailto:Martin.Adam.Stoffel@gmail.com}{\nolinkurl{Martin.Adam.Stoffel@gmail.com}}

\address{%
Joseph I. Hoffman\\
Department of Animal Behaviour\\
Bielefeld University\\ Morgenbreede 45\\ 33615 Bielefeld\\
}
\href{mailto:j_i_hoffman@hotmail.com}{\nolinkurl{j\_i\_hoffman@hotmail.com}}

